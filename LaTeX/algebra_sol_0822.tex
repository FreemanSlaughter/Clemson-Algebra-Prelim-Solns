
\begin{itemize}

\item[1.] Let $X$ be a finite-dimensional $\bbc$-vector space. A linear map $P:X \rightarrow X$ is a \textit{projection} (not necessarily orthogonal) if $P^2 = P$.
\begin{enumerate}[(a)]
\item Show that if $P$ is a projection, then there are complementary subspaces $Y$ and $Z$ such that 
$$X=Y\oplus Z, \ \ \ Py=y, \text{ for all } y \in Y, \ \ \ \text{ and } \ \ \ Pz=0, \text{ for all } z \in Z.$$
\begin{proof}
Write $x = (x - Px) + Px$. It's then apparent that $Y = \Kern(P)$ and $Z = \Ran(P)$ is the correct choice. To show $x - Px \in Y$, note $P(x - Px) = Px - Px = 0$. For $Y \cap Z = \{0\}$, let $x$ be in both. As $x \in Z$, there exists a $z$ such that $x = Pz$. Then 
$$x = Pz = P(Pz)) = Px = 0.$$
\end{proof}

\item Show that $Q:=I-P$ is also a projection. Find its image and nullspace.
\begin{proof}
$$Q^2 = (I-P)^2 = I - 2P - P^2 = I-P = Q.$$
It is straightforward to verify: $\Ran(Q) = \Kern(P)$ and $\Kern(Q) = \Ran(P)$.
\end{proof}

\item Show that in matrix form, $P$ is similar to the block matrix $M = \begin{bmatrix}
I & 0 \\
0 & 0
\end{bmatrix}.$
\begin{proof}
Since $P^2 = P$, taking determinants of both sides shows that $P$ has eigenvalues only 0 or 1. The dimension of the range is 1, so there will be no 1's on the superdiagonal of the Jordan block. 
\end{proof}

\item Show that the dimension of the image of $P$ is equal to its trace.
\begin{proof}
From (c) we know $P = UMU^{-1}$. Using (cyclic) commutativity of trace, $\text{tr}(P)$ is equal to the multiplicity of 1 as an eigenvalue which  is equal to $\dim(\text{Im}(P))$.
\end{proof}

\item If $X$ is an inner product space, show that $Y$ and $Z$ are orthogonal if and only if $P=P^*$, where $P^*:X \rightarrow X$ is the \textit{adjoint} of $P$.
\begin{proof}
$(\Rightarrow)$ Suppose that $Y$ and $Z$ are orthogonal. We know that $\ip{Px}{w} = \ip{x}{P^* w}$. Let $x=y_1+z_1$ and $w = y_2+z_2$.
\[\ip{P(y_1 + z_1)}{y_2 + z_2} = \ip{y_1}{y_2 + z_2} = \ip{y_1}{y_2} + \ip{y_1}{z_2} = \ip{y_1}{y_2}.\]
Similarly, the same computation reveals $\ip{y_1 + z_1}{P(y_2 + z_2)} = \ip{y_1}{y_2}$. Thus $\ip{x}{P^* w} = \ip{Px}{w} = \ip{x}{P w}$, so $P=P^*$.

$(\Leftarrow)$ Suppose $P = P^*$. Then $\ip{y}{z} = \ip{Py}{z} = \ip{y}{Pz} = 0$.
\end{proof}


\item Show that if $P=P^*$, then $||x||^2 = ||Px||^2 = ||Qx||^2$ for all $x \in X$.
\begin{proof}
$||Px||^2 = \ip{Px}{Px} = \ip{x}{Px}$, so $||x||^2 - ||Px||^2 = \ip{x}{x-Px} = 0$. Similarly with $Q = I-P$.
\end{proof}
\end{enumerate}











\item[2.] Let $S \in \bbr^{n \times n}$ and define the linear mapping $T: \bbr^{n \times n} \rightarrow \bbr^{n \times n}$ by $T(P) = PS+SP$.
\begin{enumerate}[(a)]
\item Let $\lambda$ be an eigenvalue of $S$ and let $u$ be its corresponding eigenvector such that $u \in \Null(T(P))$ and $Pu \neq 0$.
    \begin{enumerate}[(i)]
        \item Show that $Pu$ is an eigenvector of $S$.
        \begin{proof}
        $0 = T(P)(u) = PSu + SPu = \lambda Pu + S(Pu)$. Thus $S(Pu) = -\lambda (Pu)$.
        \end{proof}
        
        \item What is the corresponding eigenvalue of the eigenvector $Pu$?
        \begin{proof}
        $-\lambda$
        \end{proof}
    \end{enumerate}

\item Prove that if $S$ is symmetric and positive definite, then $T$ is injective.
\begin{proof}

\end{proof}

\item Show that if $S$ is symmetric and positive definite, then every $A \in \bbr^{n \times n}$ can be written as $A=PS+SP$ for some $P \in \bbr^{n \times n}$.
\begin{proof}
As this is a finite-dimensional vector space (see 3a below) $T$ injective implies $T$ surjective. Thus for every $A$, there exists a $P$ s.t. $A = T(P) = PS+SP$.
\end{proof}
\end{enumerate}







\item[3.] Let $\phi : V \rightarrow V$ be a linear map on a vector space $V$ over a field $\bbf$.
\begin{enumerate}[(a)]
\item Suppose $V$ is finite dimensional. Show that $\phi$ is surjective (onto) iff it is injective. 
\begin{proof}
Let $\phi : V \rar W$. Recall that $\phi$ surjective iff $\Ran(\phi) = W$, and $\phi$ injective iff $\Kern(\phi) = \{0\}$. By the Fundamental Theorem of Linear Algebra, $\dim(V) = \dim\Kern(\phi) + \dim\Ran(\phi)$. Thus if $\dim(V) = \dim(W)$, we have that $\phi$ is surjective iff $\phi$ is injective. Since $W=V$ here, the result follows.
\end{proof}

\item Suppose $V$ is infinite dimensional. Give an example where $\phi$ is surjective but not injective, and an example where $\phi$ is injective but not surjective.
\begin{proof}
The right-shit operator $S$ s.t. $S(x_1, x_2, x_3,...) = (0, x_1, x_2, ...)$  is clearly injective, but cannot be surjective. On the other hand, the left-shift operator $T$ s.t. $T(x_1, x_2, x_3,...) = (x_2, x_3, x_4, ...)$ is clearly surjective, but is not injective.
\end{proof}

\item A subspace $S$ of $V$ is called $\phi$-invariant if $\phi(s) \in S$ for each $s \in S$.
    \begin{enumerate}[(i)]
        \item Show that $\phi$ has an eigenvalue $\lambda \in \bbf$ iff $V$ has a $\phi$-invariant subspace of dimension 1.
        \begin{proof}
        $(\Rightarrow)$ Let $\phi(x) = \lambda x$. Then for any $s \in \text{Span}(x)$, we have that $\phi(s)$ is a multiple of $s$, thus still inside the span.

        $(\Leftarrow)$ 
        \end{proof}
        
        \item Construct an example where $\phi$ has no eigenvalue in $\bbf$ and $V$ is the direct sum of two nontrivial $\phi$-invariant subspaces. 
        \begin{proof}

        \end{proof}
    \end{enumerate}

\end{enumerate}





\item[4.] In the following questions, either provide an example or prove the requested statement. Make sure to explain why your example satisfies the given conditions. 
\begin{enumerate}[(a)]
\item In the following questions, suppose that $G$ is a group, $H$ a normal subgroup of $G$, and $K$ a normal subgroup of $H$.
    \begin{enumerate}[(i)]
        \item Find an example of groups $G$, $H$, and $K$ so that $K$ is not normal in $G$.
        \begin{proof}
        Let $G=D_8 = \langle r, s \, | \, r^4 = s^2 = e, srs = r^3 \rangle$. Take $H = \langle r^2, s \rangle$ and $K = \langle s \rangle$. By looking at the subgroup diagram (Macauley loves these) or computing the order, we see that $K \vartriangleleft H$ and $H \vartriangleleft G$. However, $K$ is not normal in $G$, which is easily verified.
        \end{proof}
        
        \item Prove that if $H$ is cyclic, then $K$ is normal in $G$.
        \begin{proof}
        Let $H = \langle x \rangle \vartriangleleft G$. Then any $K \subseteq H$ is of the form $\langle x^n \rangle$. So for any $g \in G$, since $H \vartriangleleft G$, we get that $g^{-1} x^k g$ is still in $H$, say it equals $x^m$. Thus
        \[g^{-1} (x^n)^k g = (g^{-1} x^k g)^n = (x^m)^n = (x^n)^m \in K.\]
        Therefore $K \vartriangleleft G$.
        \end{proof}
    \end{enumerate}


\item In the following questions, suppose that $R$ and $S$ are rings, $I$ is an ideal of $R$, and $\varphi : R \rightarrow S$ is a ring homomorphism. 
    \begin{enumerate}[(i)]
        \item Find an example of rings $R$ and $S$ and ideal $I$ where the image $\varphi(I)$ of $I$ is not an ideal of $S$.
        \begin{proof}
        Consider the inclusion map $\iota : \bbz \rar \bbq$. Of course, $\bbz$ is an ideal of itself, but not $\bbq$ since $2 \in \bbz$ times $\frac{1}{3} \in \bbq$ is not in $\bbz$.

        In fact, many inclusion maps give the same type of counterexample: for $\phi : \bbz \rar \bbz[x]$, the ideal $2\bbz$ maps to a non-ideal.
        \end{proof}
        
        \item Prove that if $\varphi$ is surjective, then the image $\varphi(I)$ of $I$ is an ideal of $S$.
        \begin{proof}
        Suppose $\varphi$ is surjective. Trivially, $\varphi(0) \in \varphi(I)$.
        Let $x,y \in \varphi(I)$ s.t. $x = \varphi(a)$ and $y = \varphi(b)$ for some $a,b \in I$. Then
        \[x-y = \varphi(a) - \varphi(b) = \varphi(a-b) \in \varphi(I).\]
        By surjectivity, for any $s \in S$ there exists some $r \in R$ s.t. $s = \varphi(r)$. Then
        \[sx = \varphi(r) \varphi(a) = \varphi(ar) \in \varphi(I).\]
        Likewise, $xs \in \varphi(I)$, hence $\varphi(I)$ is an ideal of $S$.
        \end{proof}
        
        \item Suppose that $I$ is a prime ideal and $\varphi$ is surjective. Prove that if $\Kern(\varphi) \subseteq I$, then the image of $I$ is a prime ideal of $S$.
        \begin{proof}

        \end{proof}
    \end{enumerate}
\end{enumerate}






\item[5.] For this problem, $R$ will be a commutative ring with identity. We say that a (proper) ideal $I \subset R$ is \textit{primary} if whenever $ab \in I$ and $a \not\in I$ then $b^n \in I$ for some $n \in \bbn$. Additionally, if $I \subseteq R$ is an ideal, we recall that the \textit{radical} of $I$ is $\sqrt{I} = \{x \in R | x^n \in I \text{ for some } n \in \bbn\}$. If $I = \sqrt{I}$, we say that $I$ is a radical ideal.
\begin{enumerate}[(a)]
\item Show that the ideal $I$ is prime if and only if it is both primary and radical.
\begin{proof}
$(\Rightarrow)$ Clearly, a prime ideal is primary. Also, it's easy to see that for a prime idea $I$, we have $I \subseteq \sqrt{I}$. To see the opposite inclusion, let $a \in \sqrt{I}$ implying that $a^k \in I$ for some $k$. Then either $a \in I$ or $a^{k-1} \in I$. If $a \in I$, we're done. If $a^{k-1} \in I$, the either $a \in I$ or $a^{k-2} \in I$...

Continuing this, we get that $a \in I$, hence $I = \sqrt{I}$.

$(\Leftarrow)$

\end{proof}

\item Characterize the radical of a primary ideal.
\begin{proof}
Let $P$ be a primary ideal. Suppose $ab \in \sqrt{P}$, implying there exists an $k$ s.t. $a^k \cdot b^k = (ab)^k \in P$. As $P$ is primary, thus either $a^k \in P$ or $(b^k)^n \in P$ for some $n$. By definition of a radical ideal, thus either $a$ or $b$ is in $\sqrt{P}$, thus $\sqrt{P}$ is a prime ideal.
\end{proof}

\item Suppose that $R$ is a PID. Characterize the nonzero primary ideals.
\begin{proof}
%From (e), we can guess that in a PID, $P$ is primary iff $P = (p^n)$ for $p \in P$. The idea is to peel off prime factors.
\end{proof}

\item Show that $I$ is primary if and only if all zero-divisors in $R/I$ are nilpotent.
\begin{proof}

\end{proof}

\item Give an example to show that, in general, there are primary ideals which are not prime powers. 
\begin{proof}
A classic example is $\mathbb{F}[x, \, y]$, with the ideal $I = (x, \, y^2)$. Then $I$ is a primary ideal, but not a power of $(x, \, y)$ - thus is not a power of a prime ideal (even in a Noetherian ring).

\medskip 
\textit{Remark}: A classic converse is $\mathbb{Z}[x]$ s.t. the coefficient on $x$ is divisible by $3$. Then $J = (3x, \, x^2, \, x^3)$ is a prime ideal, but $J^2$ is not primary. So this shows powers of a prime ideal may not be primary. 

\end{proof}
\end{enumerate}










\item[6.] For this problem, let $G$ be a finite group of order $132 = 2^2 \cdot 3 \cdot 11$.
\begin{enumerate}[(a)]
\item For each prime, $p$, dividing the order of $G$, enumerate the possibilities for the number of Sylow $p$-subgroups of $G$.
\begin{proof}

\end{proof}

\item Show that $G$ cannot be simple.
\begin{proof}

\end{proof}

\item Let $P$ be a Sylow 11-subgroup of $G$. show that if $P$ is not normal in $G$, then $N_G (P) = P$.
\begin{proof}

\end{proof}

\item Use the previous part to show that if $P$ is not normal in $G$ then all elements of $G$ that are not contained in some Sylow 11-subgroup of $G$ must be conjugate.
\begin{proof}

\end{proof}

\item Show that $G$ has a unique (normal) Sylow 11-subgroup.
\begin{proof}

\end{proof}
\end{enumerate}












\end{itemize}