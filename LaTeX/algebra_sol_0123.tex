
\begin{itemize}

\item[1.]
\begin{enumerate}[(a)]
    \item Let $A,B \in \bbc^{n \times n}$ be Hermitian matrices. Prove that the eigenvalues of
    $(A + B)$ are real.
    \begin{proof}

    \end{proof}
    
    \item Let $A,B \in \bbr^{n \times n}$ be symmetric matrices. State and prove a sufficient condition about the matrices $A, B$ in order to conclude that AB has only real eigenvalues
    \begin{proof}
    
    \end{proof}
    
    \item Let $A \in \bbc^{n \times n}$ be an arbitrary matrix. Prove that if $\ip{Ax}{x} = 0$ for all $x \in \bbc^{n}$, then $A = 0$.
    \begin{proof}
    
    \end{proof}
    
    \item Let $A \in \bbr^{n \times n}$ be an arbitrary matrix. Prove or disprove: if $\ip{Ax}{x} = 0$ for all $x \in \bbr^{n}$, then $A = 0$.
    \begin{proof}
    
    \end{proof}
\end{enumerate}





\item[2.]
\begin{enumerate}[(a)]
    \item Let $T : V \rar V$ be a linear operator. Suppose $v_1, ..., v_n$ are non-zero vectors in $V$ such that $T(v_1) = 0$ and $T(v_i) = v_{i-1}$ for $2 \leq i \leq n$. Prove that $\{v_1, ..., v_n\}$ is a linearly independent set.
    \begin{proof}

    \end{proof}

    \item Let $B = \{u_1, . . . , u_n\}$ be a basis of a vector space $V$. Let $C =\{v_1, . . . , v_m\}$ be a linearly independent set in $V$. Prove that there is an integer $k$, $1 \leq k \leq n$, such that vectors $u_k, v_2, . . . , v_m$ are linearly independent.
    \begin{proof}

    \end{proof}

    \item Let $V$ be the vector space of all polynomial functions from $\bbr$ to $\bbr$ which have degree less than or equal to $n - 1$. Let $t_1, . . . , t_n$ be any $n$ distinct real numbers, and define linear functionals $L_i(p) = p(t_i)$ on $V$. Show that $L_1, . . . , L_n$ are linearly independent.
    \begin{proof}

    \end{proof}
\end{enumerate}






\item[3.] Let $\bbf$ be a field and $V$ a vector space over $\bbf$ of dimension $n$. Let $\phi : V \rar V$ be an
$\bbf$-linear map. For any polynomial $g(x) \in \bbf[x]$, define
$$B(g) = \{v \in V : g(\phi)(v) = 0\}.$$
Note that $B(g)$ is a subspace of $V$.
\begin{enumerate}[(a)]
    \item For any $g(x) \in \bbf[x]$, prove that $B(g)$ is $\phi$-invariant.
    \begin{proof}

    \end{proof}

    \item Let $g(x) \in \bbf[x]$ be the minimal polynomial of $\phi$. Suppose $g = g_1(x)g_2(x) ... g_t(x)$ where $g_1(x), ..., g_t(x) \in \bbf[x]$ are pairwise relatively prime. Prove that $$V = B(g_1) \oplus ... \oplus B(g_t),$$ that is, for every $v\in V$, there exist unique $v_1 \in B(g_1), ..., v_t \in B(g_t)$ so that $v = v_1 + ... + v_t$.
    \begin{proof}

    \end{proof}

    \item Suppose $g(x) \in \bbf[x]$ is irreducible of degree $m$ and $t \geq 1$ is an integer. Give an example of $V$ with dimension $n = mt$ and a linear map $\phi$ so that $\phi$ has characteristic polynomial $g(x)^t$ and minimal polynomial $g(x)$. In your example with $m > 1$, explain why $\phi$ has no eigenvector in $V$. 
    \begin{proof}

    \end{proof}
\end{enumerate}





\item[4.] Let $G$ be a group of order $n = 2km$ where $m$ is odd and $k \geq 1$. Suppose that $G = {g_1, ..., g_n}$. If you do not know how to do one part, you may earn points on subsequent parts by assuming the previous results.
\begin{enumerate}[(a)]
    \item Define a map $\phi : G \rar S_n$ where $h \mapsto \pi_h$ with the property that $g_{\pi_h(i)} = hg_i$. Prove that $\phi$ is an injective homomorphism.
    \begin{proof}

    \end{proof}

    \item Prove that $\pi^t_{h}(i) = i$ for some $i$ if and only if $|h|$ divides $t$. Here, $\pi^t_{h}$ denotes applying $\pi_{h}$ $t$-many times. From this, conclude that $\pi_{h}$ is a product of cycles of length $|h|$.
    \begin{proof}

    \end{proof}

    \item Prove that the sign of the permutation $\pi_{h}$ is $(-1)^{\frac{|G|}{|h|}}$.
    \begin{proof}

    \end{proof}

    \item Prove that $\pi_{h}$ is an odd permutation if and only if $2^k$ divides $|h|$. Conclude that $\phi(G)$ has an odd permutation if and only if $G$ has an element whose order is divisible by $2^k$.
    \begin{proof}

    \end{proof}

    \item Suppose that $G$ has an element whose order is divisible by $2^k$. Prove that $\phi(G)A_n = S_n$. Conclude that $\phi(G) \cap A_n$ has index $2$ in $\phi(G)$, and, hence, that $G$ is not simple.
    \begin{proof}

    \end{proof}
\end{enumerate}





\item[5.] Given a group $G$, let $\text{Aut}(G)$ be its automorphism group
\begin{enumerate}[(a)]
    \item For some fixed $g \in G$, define the function $$\varphi_g : G \rar G, \ \ \  \varphi_g : x \mapsto g^{-1}xg.$$ Show that $\varphi_g$ is an automorphism.
    \begin{proof}

    \end{proof}

    \item Automorphisms of the form $\varphi_g$, as defined in Part (a), are called inner automorphisms. Let $$\text{Inn}(G) = \{ \varphi_g | g \in G \}$$ be the set of inner automorphisms. Show that $\text{Inn}(G)$ is a subgroup of $\text{Aut}(G)$ and that it is normal.
    \begin{proof}

    \end{proof}

    \item Show that the map $G \rar \text{Inn}(G)$ defined by $g \mapsto \varphi_g$ is a homomorphism.
    \begin{proof}

    \end{proof}

    \item Show that $\text{Inn}(G) \cong G/\text{Z}(G)$, where $\text{Z}(G)$ is the center of $G$.
    \begin{proof}

    \end{proof}

    \item Prove that if $G/\text{Z}(G)$ is cyclic, then $G$ is abelian.
    \begin{proof}

    \end{proof}

    \item Show that $\text{Inn}(G)$ cannot be a nontrivial cyclic group.
    \begin{proof}

    \end{proof}
\end{enumerate}





\item[6.] In this problem, you will prove the four isomorphism theorems for rings. Rings
are additive abelian groups, and you can and should assume all results from group
theory, such as the group isomorphism theorems, and the specific homomorphism(s)
used to prove them, which will be described below. A key aspect of this problem
is recognizing and understanding what you have to prove; none of the individual
parts should be long.
\begin{enumerate}[(a)]
    \item Show that the kernel of a ring homomorphism is a $2$-sided ideal.
    \begin{proof}

    \end{proof}

    \item The \textit{fundamental homomorphism theorem} (FHT) says that if $\phi : R \rar S$ is a ring homomorphism, then $R/\text{Ker}(\phi) \cong \text{Im}(\phi) $. The proof of this for groups involves constructing a map $$\iota : R/I \rar \text{Im}(\phi), \ \ \  \iota(r + I) = \phi(r),$$ where $I = \text{Ker}(\phi)$, and showing that it is a well-defined group isomorphism. Carry out the remaining details to prove the ring-theoretic version of the FHT.
    \begin{proof}

    \end{proof}

    \item By the \textit{correspondence theorem}, every subgroup of $R/I$ has the form $S/I$ for some subgroup $S$ satisfying $I \leq S \leq R$.
        \begin{enumerate}[i]
          \item Show that $S/I$ is a subring of $R/I$ if and only if $S$ is an subring of $R$.
           \begin{proof}

           \end{proof}

        \item Show that $J/I $is an ideal of $R/I$ if and only if $J$ is an ideal of $R$.
           \begin{proof}

           \end{proof}
        \end{enumerate}
    

    \item The \textit{fraction theorem} says that $(R/I)/(J/I) \cong R/J$. The proof of this theorem for groups involves showing that the following map $$\phi : R/I \rar R/J, \ \ \ \phi(r + I) = r + J$$ is a group homomorphism with $\text{Ker}(\phi) = J/I$. Carry out the remaining details to establish the ring-theoretic version of this theorem.
    \begin{proof}

    \end{proof}

    \item The \textit{diamond theorem} says that $(S + I)/I \cong S/(S \cap I)$ for a subring $S$ and ideal $I$.
    \begin{enumerate}[i]
          \item Show that the subgroup $S \cap I$ of $S$ is also an ideal.
           \begin{proof}

           \end{proof}

        \item In proving the diamond theorem for groups, the map $$\phi : S \rar (S + I)/I, \ \ \ \phi(s) = s + I$$ is shown to be a group homomorphism with $\text{Ker}(\phi) = S \cap I$. Carry out the remaining details to prove the ring-theoretic version.
           \begin{proof}

           \end{proof}
        \end{enumerate}
\end{enumerate}




\end{itemize}