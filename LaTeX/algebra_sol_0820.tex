
\begin{itemize}

\item[1.] Let $V$ be a vector space over a field $k$, and consider the following subset of the
vector space $V \times V$.
$$\Delta = \{(v,v) \ | \ v \in V\}.$$

\begin{enumerate}[(a)]
    \item Prove that $\Delta$ is a subspace of $V \times V$.
    \begin{proof}

    \end{proof}
    
    \item Prove that $\Delta \cong V \cong (V \times V)/\Delta$. Do not assume that $V$ is finite dimensional over $k$.
    \begin{proof}

    \end{proof}
    
    \item Give another proof of part (b) assuming that $V$ is finite dimensional over $k$.
    \begin{proof}

    \end{proof}

Assume now that $V$ is a vector space over $\bbr$ equipped with a Euclidean structure denoted $\ip{\cdot}{\cdot}$.

    \item Prove that the formula $\langle(v, w),(v',w')\rangle = \langle v,v' \rangle + \langle w,w' \rangle$ describes a Euclidean structure on $V \times V$.
    \begin{proof}

    \end{proof}
    
    \item Prove that $(v, v) \perp (-w, w)$ for all $v, w \in V$.
    \begin{proof}

    \end{proof}
    
    \item Prove that $\Delta^{\perp} \cong V$ Do not assume that $V$ is finite dimensional over $k$.
    \begin{proof}

    \end{proof}
    
    \item Give another proof of part (f) assuming that $V$ is finite dimensional over $k$.
    \begin{proof}

    \end{proof}
\end{enumerate}







\item[2.] We consider the inner product space $\bbc^n$ with its standard inner product$ \ip{u}{v} = u_1v_1 + ... + u_nv_n$. Let $T : \bbc^n \rar \bbc^n$ be defined by
$$T(z_1, z_2, ..., z_n) = (z_2-z_1, z_3-z_2, ..., z_1-z_n).$$
\begin{enumerate}[(a)]
    \item Give an explicit expression for the adjoint, $T^{*}$. Justify your answer.
    \begin{proof}

    \end{proof}
    
    \item Find the characteristic and the minimal polynomials of $T$. Is $T$ diagonalizable? Explain.
    \begin{proof}

    \end{proof}
    
    \item Does $\bbc^n$ have an orthonormal basis of eigenvectors for $T$? Justify your answer.
    \begin{proof}

    \end{proof}
    
    \item Prove that $\bbc^n$ is an orthogonal direct sum of the range of $T$ and the kernel of $T$, i.e., $\bbc^n = R(T) \ \dot{\oplus} \ K(T)$.
    \begin{proof}

    \end{proof}
    
    \item Is it true that for every linear map $S : \bbc^n \rar \bbc^n$, $\bbc^n = R(T) \ \dot{\oplus} \ K(T)$? Prove or disprove.
    \begin{proof}

    \end{proof}
\end{enumerate}







\item[3.] Let $\bbf$ a field and $V$ a vector space over $\bbf$ of dimension $n$. Let $\phi : V \rar V$ be an
$\bbf$-linear map. For any polynomial $g(x) \in \bbf[x]$, define
$$V_g = \{v \in V \ : \ g(\phi)(v) = 0\}$$
\begin{enumerate}[(a)]
    \item Show that $V_g$ is a $\phi$-invariant subspace of $V$. (Need to show that it’s a linear subspace and, for each $v \in V_g$, $\phi(v) \in V_g$.)
    \begin{proof}

    \end{proof}
    
    \item Suppose $g(x), h(x) \in \bbf[x]$ are such that $g(\phi)h(\phi) = 0$, the zero map on $V$, and $\text{gcd}(g(x), h(x)) = 1$. Show that $V$ is the direct sum of $V_g$ and $V_h$.
    \begin{proof}

    \end{proof}
    
    \item Suppose $\bbf = \bbc$, and $\phi$ has characteristic polynomial $f(x) = x^m (x^2+1)^t$ where $m + 2t = n$ and minimal polynomial $m(x) = x(x^2 + 1)$. Show that $\phi$ has $n$ independent eigenvectors in $V$, hence the matrix of $\phi$ under any basis of $V$ is diagonalizable.
    \begin{proof}

    \end{proof}
    
    \item Give an example of $V$ and $\phi$ so that $\phi$ has characteristic polynomial $f(x) = x^m (x^2+1)^t$ where $m+2t = n$ and minimal polynomial $m(x) = x^2(x^2+1)$, and show that the matrix of $\phi$ under any basis of $V$ is not diagonalizable. (Hint: Fix a basis of $V$, construct the matrix of $\phi$ under this basis, using Jordan normal form.)
    \begin{proof}

    \end{proof}
\end{enumerate}








\item[4.] For this problem, let $G$ be a group, $H \leq G$ a subgroup, and $N := \bigcap_{g \in G} gHg^{-1}$.

\begin{enumerate}[(a)]
    \item Show that $N$ is a normal subgroup of $G$. 
    \begin{proof}

    \end{proof}
    
    \item Show that if $M$ is any normal subgroup of $G$ contained in $H$, then $M$ is contained in $N$.
    \begin{proof}

    \end{proof}
    
    \item Show that if $|H| = m$ and $H$ is the only subgroup of $G$ of order $m$, then $H$ is normal in $G$.
    \begin{proof}

    \end{proof}
    
    \item Is the converse to the previous part true (that is, if $H$ is of order $m$ and normal in $G$, is it true that $H$ is the only subgroup of $G$ of order $m$)? Prove this or give a counterexample.
    \begin{proof}

    \end{proof}
    
    \item Suppose that $H$ is a subgroup of $A_n$, $n \geq 5$ with $|H| = m$, and $1 < m < \frac{n!}{2}$. Show that there are at least $n$ subgroups of $A_n$ of order $m$.
    \begin{proof}

    \end{proof}
\end{enumerate}








\item[5.] Consider a left action of a group $G$ on a set $X$, both finite.

\begin{enumerate}[(a)]
    \item Fix $x \in X$, and let $H = \Stab(x)$, the stabilizer of $x$. Show that group elements $g_1$ and $g_2$ send $x$ to the same element of $X$ if and only if they are in the same left coset of $H$.
    \begin{proof}

    \end{proof}
    
    \item Recall \textit{Lagrange’s theorem}, that $|G| = [G : H]|H|$. Show how this, along with Part (a), implies the \textit{orbit-stabilizer theorem}:
    $$|G| = |\Orb(x)||\Stab(x)| \text{ for any } x \in X.$$
    \begin{proof}

    \end{proof}
    
    \item Show how the orbit-stabilizer theorem, applied to the appropriate group action, implies Cayley’s theorem: If $|G| = n$, then there is an embedding $G \hookrightarrow S_n$.
    \begin{proof}

    \end{proof}
    
    \item Show how the orbit-stabilizer theorem, applied to the appropriate group action, implies that the size of any conjugacy class of $G$ divides $|G|$.
    \begin{proof}

    \end{proof}
    
    \item Prove that if $|G| = p^n$, then $|Z(G)| > 1$, where $Z(G)$ is the center of $G$.
    \begin{proof}

    \end{proof}
\end{enumerate}










\item[2.] For this problem, we let $p$ be a positive prime integer, $\bbz$ the ring of integers, and $\bbz_n$ the ring of integers modulo the natural number $n$. If $R$ is a ring, then the notation $R[x]$ means the ring of polynomials over $R$. If $k \in Z$ then we denote its reduction modulo $n$ by $\overline{k}$.

\begin{enumerate}[(a)]
    \item Show that the ring $\bbz_p$ is a field.
    \begin{proof}

    \end{proof}
    
    \item Show that the ring $\bbz_p[x]$ is a PID.
    \begin{proof}

    \end{proof}
    
    \item Show that if $R$ is a commutative ring with identity, and $M \subseteq N \subsetneq R$ are proper ideals, then $N$ is a maximal ideal of $R$ if and only if $N/M$ is a maximal ideal of $R/M$.
    \begin{proof}

    \end{proof}
    
    \item Let $n \in \bbn$ and $f(x) = x^n + m_{n-1} x^{n-1} + ... + m_1 x + m_0 \in \bbz[x]$. Show that $(p, f(x))$ is a maximal ideal of $\bbz[x]$ if and only if $x^n + \overline{m}_{n-1} x^{n-1} + ... + \overline{m}_1 x + \overline{m}_0$ is an irreducible polynomial in $\bbz_p[x]$.
    \begin{proof}

    \end{proof}
    
    \item If $(p, f(x))$ is a maximal ideal of $\bbz[x]$, how many elements does $\bbz[x]/(p, f(x))$ have? Explain.
    \begin{proof}

    \end{proof}
\end{enumerate}













\end{itemize}