
\begin{itemize}

\item[1.] Let $\calm_n(\bbf)$ denote the set of $n \times n$ matrices over the field $\bbf$, and let $I_n$ denote the
$n \times n$ identity matrix.
\begin{enumerate}[(a)]
\item Show that if there exists a matrix $A \in \calm_n(\bbr)$ such that $A^2 = -I_n$,
then $n$ is even.
\begin{proof}
Recall that $\det(aA) = a^n \det(A)$. Thus $\det(A)^2 = \det(A^2) = \det(-I) = (-1)^n$. Thus $n$ must be even.
\end{proof}


\item Give an explicit example to show the implication from part (a) does
not hold for matrices $A \in \calm_n(\bbc)$.
\begin{proof}
Note that the $3 \times 3$ matrix $iI$ satisfies $(iI)^2 = -I$.
\end{proof}


\item Let $M \in \calm_n(\bbf)$ be nilpotent (i.e. there exists some $l \in \bbn$ such that $M^l = 0$).
    \begin{enumerate}[(i)]
        \item What can you say about the eigenvalues and characteristic polynomial of $M$?
        \begin{proof}
        The eigenvalues are all 0, and the characteristic poly looks like $x^n$.
        \end{proof}
        
        \item Show that $M + I_n$ is invertible.
        \begin{proof}
        This is a classic ring-theoretic question. Multiply by $I -M + M^2 - M^3 + ... \pm M^n$.
        \end{proof}
        
        \item Show that $M^n = 0$ (Recall, $M$ is an $n \times n$ matrix).
        \begin{proof}
        This can be shown by considering that the eigenvalues are all 0.
        \end{proof}
    \end{enumerate}


\item Let $A,B \in \calm_n(\bbc)$ such that $AB = BA$, and $A$ has $n$ distinct eigenvalues. Show that $A$ and $B$ share at least one common eigenvector.
\begin{proof}
This question is asked often, but the ``$n$ distinct eigenvalues" part makes it easier.

Suppose $Ax = \lambda x$. Then $A(Bx) = B(Ax) = \lambda (Bx)$ implies that both $x$ and $Bx$ are eigenvectors of A. But wait a minute - since A has $n$ distinct eigenvalues, its eigenspaces are 1-dim. Thus $Bx$ is a multiple of $x$, so $x$ is an eigenvalue of both $A$ and $B$.
\end{proof}
\end{enumerate}











\item[2.] Let $T : X \rar X$ be a linear map on an $n$-dimensional vector space, and let $R_T$ and
$N_T$ denote its range and nullspace, respectively.
\begin{enumerate}[(a)]
\item Show that if $T^2 = T$, then $X = R_T \oplus N_T$. That is, every $x \in X$ can
be written uniquely as the sum of an element in the range and an element in
the nullspace.
\begin{proof}
    For all $x \in X$, write $x = (x-Tx) + Tx$. Applying $T$ to this first part gives $T(x-Tx) = Tx - Tx = 0$, so $x-Tx \in N_T$ and $Tx \in R_T$. Now let $x \in N_T \cap R_T$. Then $Tx = 0$, and there exists a $y$ s.t. $x = Ty$. Then $0 = Tx = T^2 y = Ty = x$, so this sum is direct.
\end{proof}


\item Give an explicit example to show how the result to Part (a) can fail if
$T^2 \neq T$. For the remainder of this problem, do \textit{not} assume that $T^2 = T $.
\begin{proof}
    Consider the linear map $T(x,y) = (y,0)$. This is a linear map, where $N_T = R_T = (x,0)$. Thus this sum is not direct. 
\end{proof}


\item Show that $\dim N_{T^{i+1}} \geq \dim N_{T^i}$ and $\dim R_{T^{i+1}} \leq \dim R_{T^i}$ for each
$i \in \bbn$.
\begin{proof}
    Let $x \in N_T$. The $T^2x = T(Tx) = 0$, so $x \in N_{T^2}$. This idea can be continued.
\end{proof}

\item Show that $N_{T^{n}} = N_{T^{n+1}}$ and $R_{T^{n}} = R_{T^{n+1}}$.
\begin{proof}
    By the above, we can form the chain $N_T \leq N_{T^2} \leq ... \leq N_{T^n} \leq ...$ By finite dimension, this chain must terminate, so everything after $N_{T^n}$ is stabilized.
\end{proof}


\item Show that $X = R_{T^{n}} \oplus N_{T^{n}}$.
\begin{proof}

\end{proof}


\item Show that there exists a linear map $S : X \rar X$ such that $ST = TS$ and $ST^{n+1} = T^n$:

\todo{CREATE DIAGRAM}
\begin{proof}

\end{proof}
\end{enumerate}








\item[3.] Let $f : \bbr^4 \rar \bbr^3$ be the linear map defined by $f(x) = M x$ for $x \in \bbr^4$ where
$M = ABC$ and
$$A = \begin{pmatrix}
    1 & 1 & 1 \\
    1 & 1 & -1 \\
    1 & -2 & 0
\end{pmatrix}, \quad 
B = \begin{pmatrix}
    \frac{-1}{\sqrt{3}} & 0 & 0 & 0 \\
    0 & \frac{1}{\sqrt{6}} & 0 & 0 \\
    0 & 0 &-\sqrt{2} & 0 
\end{pmatrix}, \quad 
C = \begin{pmatrix}
    1 & 1 & 1 & 1 \\
    1 & 1 & -1 & -1 \\
    1 & -1 & 1 & -1 \\
    1 & -1 & -1 & 1
\end{pmatrix}.$$

\begin{enumerate}[(a)]
\item Define the adjoint map $f^* : \bbr^3 \rar \bbr^4$ (under the standard Euclidean
inner product) and express it in terms of $M$.
\begin{proof}
    Here, $f^*$ should just be the transpose of $M$.
\end{proof}


\item Find a singular value decomposition (SVD) of $M$. (Hint: Observe that
$A^T A$ and $C^T C$ are diagonal.)
\begin{proof}

\end{proof}


\item Find all $x \in \bbr^4$ with $\ns{x} = 1$ so that $\ns{Mx}$ is maximized.
\begin{proof}

\end{proof}

\item Describe the eigenvalues and eigenvectors of $M^T M$.
\begin{proof}
        
\end{proof}


\item Find the least square solution for $M x = b$ with $\ns{x}$ minimal where
$b = (1, 1, 1)^T$. (Hint: Use the pseudo-inverse of $M$.)
\begin{proof}

\end{proof}


\end{enumerate}





\item[4.] Let $G$ be a group with subgroups $H$ and $K$. Recall that their \textit{commutator subgroup}
is the subgroup of $G$ generated by all commutators of elements of $H$ and $K$, i.e., 
$$[H,K] = \langle [h,k] : h \in H, k \in K \rangle.$$
In this problem, we define the commutator $[h, k]$ as $hkh^{-1}k^{-1}$. A subgroup H is
called \textit{perfect} if its commutator subgroup is itself, i.e., $H = [H, H]$. Recall that the
\textit{center} of a group is the subgroup $Z(G)$ consisting of the elements that commute
with every element of $G$. In the following, you may use a previous part (even if you
can’t prove it) in subsequent parts.

\begin{enumerate}[(a)]
\item \begin{enumerate}[(i)]
        \item Prove that $[Z(G), G]$ is the trivial subgroup.
        \begin{proof}
        
        \end{proof}
        
        \item Prove that $[H, K] = [K, H]$.
        \begin{proof}

        \end{proof}
    \end{enumerate}



\item Let $q : G \rar G/Z(G)$ be the canonical quotient map. Let $Z_2$ be the preimage
in $G$ of $Z(G/Z(G))$ by $q$.
    \begin{enumerate}[(i)]
        \item Prove that $[Z_2, G] \subseteq Z(G)$.
        \begin{proof}
        
        \end{proof}
        
        \item Using Part (a), conclude that $[[Z_2, G], G]$ is the trivial group.
        \begin{proof}

        \end{proof}
    \end{enumerate}

\item Let $H$, $K$, and $L$ be subgroups of $G$.
    \begin{enumerate}[(i)]
        \item Let $h \in H$, $k \in K$, and $l \in L$. Prove that
        $$k[[k^{-1}, h], l]k^{-1}l[[l^{-1}, k], h]l^{-1}h[[h^{-1}, l], k]h^{-1}$$
        is trivial.
        \begin{proof}
        Denote by $[x,y,z] = [[x, y], y]$ and $x^y = y x y^{-1}$. Then the above looks like
        $$[k^{-1},h,l]^k \ [l^{-1},k,h]^l \ [h^{-1},l,k]^h.$$
        This is the standard form for the Hall-Witt identity. Expand: $[k^{-1},h,l]^k = [k^{-1} h k h^{-1}, l]^k =  (k^{-1} h k h^{-1} l h k^{-1} h^{-1} k l^{-1})^k = h k h^{-1} l h k^{-1} h^{-1} k l^{-1} k^{-1}$. Let $x = h k h^{-1} l h$, cycle it by one and define $y = k l k^{-1} h k$, then cycle again and define $z = l h l^{-1} k l$. Then we note that $[k^{-1},h,l]^k = xy^{-1}$. Similarly, $[l^{-1},k,h]^l = yz^{-1}$ and $[h^{-1},l,k]^h = zx^{-1}$. Thus, $[k^{-1},h,l]^k \ [l^{-1},k,h]^l \ [h^{-1},l,k]^h = xy^{-1} \ yz^{-1} \ zx^{-1}$, which is trivial.
        \end{proof}
        
        \item Suppose that $[[K, H], L]$ and $[[L, K], H]$ are both trivial. Conclude that $[[H, L], K]$ is also trivial.
        \begin{proof}
            This can be viewed as the Jacobi identity for commutators. Three subgroups lemma! Regardless, it's trivial using the above: if $[k,h,l] = e$, then $[k,h,l]^k = e$ also. Since $[k,h,l]$ and $[l,k,h]$ are trivial, we get that $[h,l,k]$ is trivial too. Thus $[h,l]$ is in the centralizer $C(K)$ for all $h$ and $l$. Since these generate $[H,L]$, we see $[H,L] \subseteq C(K)$, hence $[H,L,K]$ is trivial.
        \end{proof}
    \end{enumerate}




\item Suppose that $G$ is perfect.
    \begin{enumerate}[(i)]
        \item Prove that $[G, Z_2]$ = $[[G, G], Z_2]$.
        \begin{proof}
        
        \end{proof}
        
        \item Using Parts (a), (c), and (d), prove that $[G, Z_2]$ is trivial.
        \begin{proof}

        \end{proof}

        \item Conclude that $Z_2 = Z(G)$ and that $Z(G/Z(G))$ is trivial.
        \begin{proof}

        \end{proof}
    \end{enumerate}
\end{enumerate}







\item[5.] Fix an action $\phi$ of a group $G$ on a set $S$. A bijection $f : S \rar S$ is $G$-\textit{equivariant} if
it commutes with the action of the group. Specifically, if the action is defined by
a homomorphism $\phi : G \rar \text{Perm}(S)$ to the group of permutations of $S$, then $f$ is
$G$-equivariant if the following diagram commutes, for all $g \in G$:

\todo{ADD DIAGRAM}

Given $\phi$, denote the set of $G$-equivariant bijections of $S$ by $\text{Eq}_G(S)$.

\begin{enumerate}[(a)]
\item Show that $\text{Eq}_G(S)$ is a group.
\begin{proof}
        
\end{proof}


\item Consider the action of $G$ on the set $S$ of right cosets of some subgroup
$H \leq G$ by multiplication. Show that every $f \in \text{Eq}_G(S)$ is completely determined by the image of $H$. Specifically, if $f : H \mapsto Hx$, then $f : Hg \mapsto Hxg$,
for every $g \in G$.
\begin{proof}

\end{proof}


\item Show that if $x \in N_G(H)$, the normalizer of $H$, then $Hg \mapsto Hxg$ is a
well-defined bijection on right cosets.
\begin{proof}

\end{proof}


\item Show that $Hg \mapsto Hxg$ is $G$-equivariant if and only if $x \in N_G(H)$.
\begin{proof}
        
\end{proof}


\item Show that $x, y \in N_G(H)$ define the same equivariant bijection if and
only if they are in the same right coset of $H$.
\begin{proof}

\end{proof}


\item Show that $\text{Eq}_G(S) \cong N_G(H)/H$.
\begin{proof}

\end{proof}


\end{enumerate}






\item[6.] For this problem, we recall that a \textit{Euclidean domain} is an integral domain $R$
equipped with a function $\phi : R \ \{0\} \rar \bbn_0$ such that for all nonzero $x, y \in R$
we have $\phi(x) \leq \phi(xy)$, and for all $a, b \in R$ with $b \neq 0$ there exist $q, r \in R$ such that
$a = qb + r$ with $r = 0$ or $\phi(r) < \phi(b)$. We begin with a problem concerning groups.
\begin{enumerate}[(a)]
\item Let $G$ be an infinite group. Show that the following conditions are
equivalent
    \begin{enumerate}[(i)]
        \item Every nonidentity subgroup of $G$ is isomorphic to $G$.
        \begin{proof}
        
        \end{proof}
        
        \item Every subgroup of $G$ is cyclic.
        \begin{proof}

        \end{proof}
        \item $G \cong \bbz$.
        \begin{proof}

        \end{proof}
    \end{enumerate}


\item Show that $\bbz$ is a PID.
\begin{proof}

\end{proof}


\item More generally, show that any Euclidean domain is a PID.
\begin{proof}

\end{proof}

\item Show that if $R$ is an integral domain, then $R[x]$ is a PID if and only if $R$ is a field.
\begin{proof}

\end{proof}


\item Show that if $R$ is not a PID, then there is an ideal $P \subset R$ that is
maximal with respect to being nonprincipal.
\begin{proof}

\end{proof}


\item Show that $R$ is a PID if and only if every prime ideal is principal
\begin{proof}

\end{proof}
\end{enumerate}









\end{itemize}