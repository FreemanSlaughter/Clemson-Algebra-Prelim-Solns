
\begin{itemize}

\item[1.] Let $A: X \rar X$ be a linear map of an $n$-dimensional vector space over $\bbc$.
\begin{enumerate}[(a)]
\item Take any $v \neq 0$ in $X$  and consider the set $v, Av, A^2v, ... , A^nv$, which
must be linearly dependent. Use this to show that $A$ must have an eigenvector
and eigenvalue pair.
\begin{proof}

\end{proof}

\item Let $\lambda$ be an eigenvalue of $A$, and for each $j \in \bbn$, let $N_j$ be the
nullspace of $(A - \lambda I)^j$. Show that $A - \lambda I$ extends to a well-defined map $N_{j+1}/N_j \rar N_j/N_{j-1}$ that is injective.
\begin{proof}

\end{proof}

\item Give an example of a linear map on a 5-dimensional vector space for
which $N_5 \supsetneq N_4$ holds for some eigenvalue $\lambda$. Writing your answer in matrix
form is sufficient.
\begin{proof}

\end{proof}

\item A linear map is \textit{nilpotent} if $A^k = 0$ for some $k \in \bbn$. Suppose that $A$ is
a nilpotent map on a 4-dimensional vector space. Find all eigenvalues $\lambda$, and
for each one, list all possible sequences $(d_1, d_2, . . .)$, where $d_j = dim(N_j )$.
\begin{proof}

\end{proof}

\item A linear map is \textit{idempotent} if $A^2 = A$. Suppose that $A$ is an idempotent
map on a 4-dimensional vector space. Find all possible eigenvalues $\lambda$, and
for each one, list all possible sequences $(d_1, d_2, . . .)$, where $d_j = dim(N_j )$.
\begin{proof}

\end{proof}
\end{enumerate}






\item[2.] Let $\bbf$ be a field and $V$ a vector space over $\bbf$ of dimension $n$. Let $\phi : V \rar V$ be
an $\bbf$-linear map. For any polynomial $g(x) \in \bbf[x]$, define
$$V(g) = \{ v \in V : g(\phi)(v) = 0\}.$$
\begin{enumerate}[(a)]
\item Show that $V (g)$ is a subspace of $V$ and that it is $\phi$-invariant.
\begin{proof}

\end{proof}

\item Suppose $g(x), h(x) \in \bbf[x]$ satisfy $\text{gcd}(g(x), h(x)) = 1$. Show that $V (gh) = V (g) \oplus V (h)$.
\begin{proof}

\end{proof}

\item Suppose $g(x) \in \bbf[x]$ is a polynomial of degree $m$ and $t \geq 1$ is an
integer. Give an example of $V$ with dimension $n = mt$ and a linear map $\phi$ so
that $V (g^2) = V$ but $V (g) \neq V$.
\begin{proof}

\end{proof}

\item Suppose $g(x) \in \bbf[x]$ is irreducible of degree $m$ and $t \geq 1$ is an
integer. Give an example of $V$ with dimension $n = mt$ and a linear map $\phi$ so
that has characteristic polynomial $g(x)^t$ and minimal polynomial $g(x)$. In your example with $m > 1$, explain why $\phi$ has no eigenvector in $V$.
\begin{proof}

\end{proof}
\end{enumerate}





\item[3.] Let $V$ be a finite-dimensional inner product space over $\bbc$.
\begin{enumerate}[(a)]
\item Prove that for a linear map $T : V \rar V$ we have that for all $x, y \in V$
$$\ip{Tx}{y} = \frac{1}{4} \sum_{k=0}^3 i^k \ip{T(x+i^k y)}{x+i^k y}$$
and use this to show that if $\ip{Tv}{v} \in \bbr$ for every $v \in V$, then $T$ is self-adjoint.
\begin{proof}
    This is a standard analytic result, the Polarization Identity. Collect like terms, and simple algebra can show it. Now, if $\ip{Tv}{v} \in \bbr$, then so is the ``complex" conjugate of its adjoint, $\ip{T^* v}{v}$. Taking the difference (and since this is over $\bbc$) we get that $T - T^* = 0$.
\end{proof}

\item A linear map $T : V \rar V$ is a contraction if $||Tv|| \leq ||v||$ for every
$v \in V$. Prove that $T$ is a contraction if and only if $I - T^{*}T$ is nonnegative (i.e. positive semidefinite).
\begin{proof}
    $||Tv|| \leq ||v||$ iff $\ip{Tx}{Tx} \leq \ip{x}{x}$ iff $0 \leq \ip{(I - T^*T)x}{x}$.
\end{proof}

\item Prove that if $T : V \rar V$ is a contraction then there exists a positive
semidefinite map $S : V \rar V$ such that $S^2 = I - T^{*}T$.
\begin{proof}
    From b) we get that $I - T^*T$ is positive semi-definite. Well, by Schur decomp., any positive semi-definite matrix has a square root which is also positive semi-definite.
\end{proof}

\item A linear map $U : X \rar X$ on a complex inner product space $X$ is
unitary if $UU^{*} = I$ and $U^{*}U = I$. Let $T$ and $S$ be as in part (c). Prove that if $S$ is the zero map then $T$ is unitary. (This is not true if $V$ is infinite-
dimensional, so make sure you emphasize where you have used that $V$ is finite-dimensional.)
\begin{proof}
    We can repeat b) to see that both $TT^*$ and $T^*T$ are $I$. I believe the finite dimensional part comes into play in c).
\end{proof}

\item Let $T$ and $S$ be as in part (c). Consider the direct product $V \times V =
\{(x, y): x, y \in V\}$ where the addition and multiplication by scalars are defined
componentwise, equipped with the inner product
$$\ip{(x_1, y_1)}{(x_2, y_2)}_{V \times V} = \ip{x_1}{x_2}_{V} \ip{y_1}{y_2}_{V}$$
Prove that $L : V \rar V \times V$ defined by $Lv = (T v, Sv)$ is an isometry.
\begin{proof}

\end{proof}
\end{enumerate}








\item[4.] For this problem, $R$ will be a commutative ring with identity, $\bbf$ will be a field,
$\bbz$ will be the (ordinary) integers, and $\bbz_p$ will denote the integers modulo $p$. Also
recall that a polynomial is said to be \textit{monic} if its leading coefficient is 1.
\begin{enumerate}[(a)]
\item Show that if $R$ is a PID (principal ideal domain) then any nonzero
prime ideal of $R$ is a maximal ideal of $R$.
\begin{proof}
    Suppose we have $I \subset J \subset R$, where $I=\langle a \rangle$ and $J=\langle b \rangle$. By containment of ideals, thus $\exists x$ s.t. $a = bx$. Since $a \in I$, since it's prime, either $b \in I$ or $x \in I$. If $b \in I$, then $I = J$. If $x \in I$, since $I$ is generated by $a$, there must be some $y$ s.t. $x = ya$. But then $a = bx = bya$; that is, $by = 1$. Since $b$ is a unit in $J$, it must be that $J = R$.
\end{proof}

\item Show that there is a one-to-one correspondence between maximal ideals
of $\bbf[x]$ and monic irreducible polynomials in $\bbf[x]$.
\begin{proof}
    Let $f(x)$ be a monic, irreducible poly. with roots $a_1, ..., a_n$ not in $\bbf$. By 1st Iso.,
    $$\bbf[x] / \langle f(x) \rangle \cong \bbf[a_1, ..., a_n] \cong \bbf[x] / \langle (x-a_1), ..., (x-a_n) \rangle,$$
    where these ideals are maximal since the quotient forms a field.
\end{proof}

\item Show that if $M \subsetneq \bbz[x]$ is a maximal ideal, then $M \bigcap \bbz = (p)$ where $p$ is some nonzero prime integer.
\begin{proof}

\end{proof}

\item Now let $0 \neq p \in \bbz$ be a fixed prime. Show that there is a one-to-
one correspondence between maximal ideals of $\bbz[x]$ that contain $p$ and monic irreducible polynomials in $\bbz_p[x]$.
\begin{proof}

\end{proof}

\item Characterize all maximal ideals of $\bbz[x]$.
\begin{proof}

\end{proof}
\end{enumerate}









\item[5.] Let $G$ be a finite group and $H$ an Abelian subgroup. Recall that the centralizer
$C(g)$ of $g \in G$ is the set of all elements of $G$ which commute with $g$, the center
$Z(G)$ is the set of all elements of $g$ that commute with all elements in $G$, and the
conjugacy class $\calo(g)$ of an element $g \in G$ is the set of all conjugates of $g$ in $G$.
In the following questions, even if you skip a part, you may use the results of it
throughout the remainder of the problem without proof.
\begin{enumerate}[(a)]
\item Compute the number of conjugacy classes of $G$ as follows:
    \begin{enumerate}[(i)]
        \item Prove that $|G|/|C(g)|$ is equal to the size of the conjugacy class of $g$.
        \begin{proof}
            Consider the action of $G$ on itself by conjugation. Then by Orbit-Stabilizer, $$|\mathcal{O}(g)| = [G : \text{Stab}(g)] = \frac{|G|}{|\text{Stab}(g)|},$$
            where under this action the stabilizer $\text{Stab}(g)$ is the same as the centralizer $C(g)$.
        \end{proof}
        
        \item Prove that $\frac{1}{|G|} \sum_{g \in G} |C(g)|$ is the number of conjugacy classes of $G$.
        \begin{proof}
            Use Burnside's lemma (or the lemma that is not Burnside's) to see 
            $$\#\{\mathcal{O}(g)\} = \frac{1}{|G|} \sum_{g \in G} |C(g)|.$$
        \end{proof}
    \end{enumerate}

\item Compute a lower bound on the number of conjugacy classes of $G$ as follows:
    \begin{enumerate}[(i)]
        \item Prove that for any $h \in H$, $|C(h)| \geq |H|$.
        \begin{proof}
            Follows from $$|C(g)| = \frac{|G|}{|\mathcal{O}(g)|}.$$
        \end{proof}
        
        \item Prove that the number of conjugacy classes of $G$ is at least $\frac{|H|^2}{|G|}$.
        \begin{proof}
            Follows from ai) and bi)
        \end{proof}
    \end{enumerate}

\item Suppose that $G$ is not Abelian, then
    \begin{enumerate}[(i)]
        \item Prove that $G/Z(G)$ cannot be cyclic.
        \begin{proof}
            Contrapositive of ``if $G/Z(G)$ is cyclic, then $G$ is Abelian."
        \end{proof}
        
        \item Using Part i, prove that $|G| \geq 4|Z(G)|$.
        \begin{proof}
            If $|G/Z(G)| \in \{1,2,3\}$ then it is cyclic. 
        \end{proof}
        
        \item Prove that for $g \not\in Z(G$), $|\calo(g)| \geq 2$.
        \begin{proof}
             By contrapositive, if $|\mathcal{O}(g)| = 1$, then $g \in Z(G)$.
        \end{proof}
        
        \item Let $k(G)$ be the number of conjugacy classes in $G$ and observe
        (you do not need to prove) that $k(G)-|Z(G)|$ is the number of nontrivial
        conjugacy classes. Use the class equation to prove that
        $$|G| \geq |Z(G)| + 2(k(G)-|Z(G)|)$$
        \begin{proof}
            Combining the above with the Class Formula, we get $$|G| = |Z(G)| + \sum_{g \in G}|\mathcal{O}(g)| \geq |Z(G)| + 2(k(G) - |Z(G)|).$$
        \end{proof}
        
        \item Using the previous parts, conclude that $\frac{5}{8}|G| \geq k(G)$.
        \begin{proof}
            $$|G| \geq 2k(G) - |Z(G)| \geq 2k(G) - \frac{1}{4} |G|.$$
            Hence, $k(G) \leq \frac{5}{8}|G|$. Note this bound can be attained, for example by $\mathcal{Q}_8$.
            \medskip  
            
            This is called the $5/8$-ths Theorem, and basically states that if the probability two random group elements commute is larger than $5/8$, then then group itself is Abelian.
        \end{proof}
    \end{enumerate}
\end{enumerate}















\item[6.] Let $G$ be a group and $H \leq G$. Throughout this problem, you may use the result
of a previous part, even if you could not prove it.
\begin{enumerate}[(a)]
\item Give a direct proof (i.e., w/o appealing to Part (b)), that if $[G : H] = 2$,
then $H \trianglelefteq G.$
    \begin{proof}
        If $H$ has index 2, then for any $g \in G - H$, we have $gH = G - H$. But likewise, it must be that $Hg = G-H$ also, hence $gH = Hg$.
    \end{proof}

\item Show that if $[G : H] = p$, where $p$ is the smallest prime dividing $|G|$,
then $G$ cannot be simple.
    \begin{proof}
        This is the Strong Cayley Theorem. See Coykendall's notes, p.39. Note that $G$ acts on $H$ by left-mult., and this action induces a hom. $\varphi : G \rar S_p$ with $\Kern(\varphi) \subseteq H$. Since $G/\Kern(\varphi)$ is isomorphic to a subgroup of $S_p$, it has order dividing $p!$, but also dividing $|G|$. Hence, $|G/\Kern(\varphi)| = p$. From this, $$|G/\Kern(\varphi)| = [G : \Kern(\varphi)] = [G : H][H: \Kern(\varphi)] = p [H: \Kern(\varphi)].$$
            Therefore $[H: \Kern(\varphi)]=1$, implying they coincide. As the kernel is normal, so is $H$.
    \end{proof}

\item Show that if $G$ has a nontrivial proper subgroup $H$ of index $[G : H] < 5$,
then $G$ cannot be simple.
    \begin{proof}
        The case of index 1 is clear, and 2 is settled above. For index 3, this action gives $\varphi : G \rar S_3$. If $|G|>6$, the kernel is normal. If $|G| \leq 6$, then it cannot be simple as the smallest non-cyclic simple group is $A_5$. Similarly, if the index is 4, $G$ can be homomorphically embedded in $S_4$. As $S_4$ is solvable, $G$ simple leads to contradiction.
    \end{proof}
    
\item Now, let $G$ be a group of order 90.
    \begin{enumerate}[(i)]
        \item Suppose that $G$ has a nonnormal Sylow 5-subgroup. Show that there is a nontrivial homomorphism $\phi: G \rar S_6$.
        \begin{proof}
            Using Sylow, if $|G|=90$ then it has 6 Sylow-5 subgroups. Let $\varphi : G \rar S_6$ be the permutation of these. 
        \end{proof}
        
        \item If $\phi(G)$ is contained in $A_6$, show that $\phi$ is not injective.
        \begin{proof}
            Any order-5 element of $G$ can be embedded into $S_6$ by a 5-cycle. These cycles are in $A_6$. By contrapositive, $\varphi$ not an embedding implies it doesn't map into $A_6$.
        \end{proof}
        
        \item Show that $G$ cannot be simple.
        \begin{proof}
            If $\Kern(\varphi) = G$, then by 1st Iso. the image is trivial. If $\Kern(\varphi)$ is trivial, then $G$ is contained in $S_6$ but $G \neq S_6$, so these cycles are in $A_6$. Hence $G \cap A_6$ is normal in $G$ by 2nd Iso. Theorem.
        \end{proof}
    \end{enumerate}
\end{enumerate}
















\end{itemize}